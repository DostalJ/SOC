\pagestyle{empty}
\begin{center}

\medskip

\vfill
{\bfseries\Large{Středoškolská odborná činnost}}

\medskip

{{\bfseries Obor: } 17. Filozofie, politologie a ostatní humanitní a společenskovědní obory}
\vfill


\vspace{20mm}

{\LARGE\bfseries \#filterbubble}

\vfill

\begin{tabular}{ll}
\bfseries Autoři: & Františka Sandroni, Slovanské gymnázium Olomouc\\
                  & Jakub Dostál, Slovanské gymnázium Olomouc\\
\noalign{\vspace{2mm}}
% \bfseries Konzultant: & Chceme nějakého?, UP Olomouc\\
%\bfseries Institute: & Department of Mathematical Analysis and Applications\\&  of Mathematics, Faculty of Science, Palack\'y University
\end{tabular}

\vfill
Olomouc 2017
\end{center}

% ##############################################################################
% ##############################################################################

\newpage
\openright

\noindent
\subsection*{Poděkování}
\noindent Neděkujeme nikomu, jsme prostě nejlepší sami o sobě


% ##############################################################################
% ##############################################################################

\newpage
%%% Page containing a legal statement
\vspace*{\stretch{8}}

\noindent
Prohlašujeme, že jsme tuto práci vypracovali samostatně a~výhradně
s~použitím citovaných pramenů, literatury a~dalších odborných zdrojů.

Prohlašujeme, že tištěná verze a elektronická verze soutěžní práce SOČ jsou
shodné.

\vspace{20mm}
\noindent
%% Place and date of signature
V~Olomouci dne \makebox[2.5cm]{\dotfill}
\hspace*{\fill}
\makebox[3cm]{\dotfill}
\hspace*{\fill}
\\
\\
\\
\makebox[5.55cm]{}
\hspace*{\fill}
\makebox[3cm]{\dotfill}
\hspace*{\fill}

\vspace*{\stretch{1}}

% ##############################################################################
% ##############################################################################

\newpage
%%% Czech and English abstracts
\nobreak\vbox to 0.49\vsize{
\setlength\parindent{0mm}
\setlength\parskip{5mm}
\newcommand{\forceindent}{\leavevmode{\parindent=1em\indent}}

\textbf{Title}: \#filterbubble\\
\textbf{Authors}: Františka Sandroni, Jakub Dost\'al\\
\textbf{School}: Slavonic grammar school Olomouc\\
% \textbf{Supervisor}:
% Tom\'aš F\"urst, Department of Mathematical Analysis and Applications of Mathematics, Faculty of Science, Palack\'y University Olomouc
\textbf{Abstract:}\\
Významným zdrojem informací v dnešní době jsou sociální sítě. Obsah, který na nich jedinec pozoruje je však ovlivňověn \textit{preferenčními algoritmy}. Ty filtrují informace a vedou jedince do situace, kdy okruh příspěvků, které pozoruje, není plně vyvážený. Tím pádem se zde vytvářejí komunity, které výrazně modifikují objem a obsah informací ovlivňující jejich členy. Tomuto fenoménu se říká \textit{filter bubble} a může vést například k samovolnému vzniku extrémistických názorů a skupin, neboť jedinci v dané bublině nemají přístup k dostatečně širokému spektru informací, ale pouze k zúženému výběru.\\
\forceindent Práce nejprve představuje informační bublinu a její positiva a negativa. Poté představujeme novou metodiku výzkumu informační bubliny na datech poskytovaných \textit{Twitterem}. Tato data jsou nejprve pečlivě vybrána a následně zpracovávána pomocí \textit{sentimentální analýzy}. Narozdíl od mnoha předešlých studií pracujeme s větším počtem subjektů a jejich konkrétním chováním na sociální síti. Funkčnost našeho modelu ukážeme na několika jednoduchých příkladech, na nichž představíme jejich kladné i záporné vlastnosti.\\
\textbf{Keywords}: filterbubble, echo chamber, social media, sentiment analysis, democracy,
\vss}


\vbox to 0.5\vsize{
\setlength\parindent{0mm}
\setlength\parskip{5mm}
\newcommand{\forceindent}{\leavevmode{\parindent=1em\indent}}

\textbf{Název práce}: \#filterbubble\\
\textbf{Autoři:} Františka Sandroni, Jakub Dost\'al\\
\textbf{Škola}: Slovanské gymnázium Olomouc\\
\textbf{Abstrakt:}\\
Významným zdrojem informací v dnešní době jsou sociální sítě. Obsah, který na nich jedinec pozoruje je však ovlivňověn \textit{preferenčními algoritmy}. Ty filtrují informace a vedou jedince do situace, kdy okruh příspěvků, které pozoruje, není plně vyvážený. Tím pádem se zde vytvářejí komunity, které výrazně modifikují objem a obsah informací ovlivňující jejich členy. Tomuto fenoménu se říká \textit{filter bubble} a může vést například k samovolnému vzniku extrémistických názorů a skupin, neboť jedinci v dané bublině nemají přístup k dostatečně širokému spektru informací, ale pouze k zúženému výběru.\\
\forceindent Práce nejprve představuje informační bublinu a její positiva a negativa. Poté představujeme novou metodiku výzkumu informační bubliny na datech poskytovaných \textit{Twitterem}. Tato data jsou nejprve pečlivě vybrána a následně zpracovávána pomocí \textit{sentimentální analýzy}. Narozdíl od mnoha předešlých studií pracujeme s větším počtem subjektů a jejich konkrétním chováním na sociální síti. Funkčnost našeho modelu ukážeme na několika jednoduchých příkladech, na nichž představíme jejich kladné i záporné vlastnosti.\\
\textbf{Klíčová slova}: informační bublina, sociální sítě, sentimentální analýza, demokracie
\vss}

% ##############################################################################
% ##############################################################################

\newpage
\openright

\pagestyle{plain}
\setcounter{page}{1}

\tableofcontents
