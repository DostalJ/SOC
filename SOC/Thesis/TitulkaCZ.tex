\pagestyle{empty}
\begin{center}

\medskip

\vfill
{\bfseries\Large{Středoškolská odborná činnost}}

\medskip

{{\bfseries Obor: } 17. Filozofie, politologie a ostatní humanitní a společenskovědní obory}
\vfill


\vspace{20mm}

{\LARGE\bfseries \#filterbubble}

\vfill

\begin{tabular}{ll}
\bfseries Autoři: & Františka Sandroni\\
                  & Jakub Dostál\\
\noalign{\vspace{2mm}}
% \bfseries Konzultant: & Chceme nějakého?, UP Olomouc\\
%\bfseries Institute: & Department of Mathematical Analysis and Applications\\&  of Mathematics, Faculty of Science, Palack\'y University
\end{tabular}

\vfill
Olomouc 2017
\end{center}

% ##############################################################################
% ##############################################################################

\newpage
\openright

\noindent
\subsection*{Poděkování}
\noindent Rádi bychom poděkovali Dott. Kvidu Sandroni za projevené pozorné připomínky během celé tvorby naší práce, každá sebemenší myšlenka nám byla cennou radou a podnětem ke zdokonalování studie. Velký dík patří také Barboře Pomykalové za grafickou úpravu. Neméně děkujeme Roswitě Chvátalové a Aleně Ambrožové, které se ochotně podílely na korektuře práce.


% ##############################################################################
% ##############################################################################

\newpage
% Page containing a legal statement
\vspace*{\stretch{8}}

\noindent
Prohlašujeme, že jsme tuto práci vypracovali samostatně a~výhradně
s~použitím citovaných pramenů, literatury a~dalších odborných zdrojů.

Prohlašujeme, že tištěná verze a elektronická verze soutěžní práce SOČ jsou
shodné.

Nemáme závažný důvod proti zpřístupňování této práce v souladu se zákonem č. 121/2000 Sb., o právu autorském, o právech souvisejících s právem autorským a o změně některých zákonů (autorský zákon) v platném znění. 

\vspace{20mm}
\noindent
%% Place and date of signature
V~Olomouci dne \makebox[2.5cm]{\dotfill}
\hspace*{\fill}
\makebox[5cm]{\dotfill}
\hspace*{\fill}
\\
\\
\\
\makebox[5.55cm]{}
\hspace*{\fill}
\makebox[5cm]{\dotfill}
\hspace*{\fill}

\vspace*{\stretch{1}}

% ##############################################################################
% ##############################################################################

\newpage
%%% Czech and English abstracts
\nobreak\vbox to 0.49\vsize{
\setlength\parindent{0mm}
\setlength\parskip{5mm}
\newcommand{\forceindent}{\leavevmode{\parindent=1em\indent}}

\textbf{Title}: \#filterbubble\\
\textbf{Authors}: Františka Sandroni, Jakub Dost\'al\\
\textbf{School}: Slavonic grammar school Olomouc\\
\textbf{Abstract:}\\
Social networks are the most common source of the news and information nowadays. But their content is affected by \textit{preferetial algorithms}, which filter informations and lead user to bubble with homogenous opinions. This phenomenon is called \textit{filter bubble} and can cause creation of radical groups.\\
\forceindent We present new way to study filter bubble in online communities. Unlike most of the related work, our approach allows us to examine those effects precisely on social networks. \textit{Sentimental analysis} is used to perform measurements of the content viewed by the \textit{Twitter} users.\\
\forceindent We present the power of our method on a several examples and show the ability of providing sociological results.\\
\textbf{Keywords}: filter bubble, social media, sentiment analysis, democracy, society
\vss}


\vbox to 0.5\vsize{
\setlength\parindent{0mm}
\setlength\parskip{5mm}
\newcommand{\forceindent}{\leavevmode{\parindent=1em\indent}}

\textbf{Název práce}: \#filterbubble\\
\textbf{Autoři:} Františka Sandroni, Jakub Dost\'al\\
\textbf{Škola}: Slovanské gymnázium Olomouc\\
\textbf{Abstrakt:}\\
Významným zdrojem informací v dnešní době jsou sociální sítě, jejich obsah je však ovlivěn \textit{preferenčními algoritmy}. Ty filtrují informace a~vedou jedince do situace, kdy je uzavřen v názorově homogenní bublině. Tomuto fenoménu se říká \textit{filter bubble} a~může zapříčinit například samovolný vznik extrémistických názorů a~skupin.\\
\forceindent V naší práci představujeme nový způsob studia informační bubliny na kon\-krét\-ních online komunitách. Na rozdíl od mnoha předešlých studií nám umožňuje výzkum přímo v místě vzniku \textit{informační bubliny}, tj. na sociálních sítích. Měření je založeno na \textit{sentimentální analýze} příspěvků viditelných studovanými uživateli na \textit{Twitteru}. To nám umožňuje odhadovat velmi přesně složení příspěvků v okolí velkého množství námi zvolených uživatelů.\\
\forceindent Na několika příkladech jsme ukázali její funkčnost a schopnost podávat relevantní sociologické výsledky.\\
\forceindent Naší prací se snažíme položit základní kámen studia \textit{informační bubliny} pomocí matematiky, založeného na skutečných datech. Snažíme se otevřít dveře dalším i analytickým pohledům na tuto problematiku.\\
\textbf{Klíčová slova}: informační bublina, sociální sítě, Twitter, sentimentální analýza, společnost
\vss}

% ##############################################################################
% ##############################################################################

\newpage
\openright

\pagestyle{plain}
\setcounter{page}{1}

\tableofcontents
