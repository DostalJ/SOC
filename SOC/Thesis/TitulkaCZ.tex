\pagestyle{empty}
\begin{center}

\medskip

\vfill
{\bfseries\Large{Středoškolská odborná činnost}}

\medskip

{{\bfseries Obor: } 17. Filozofie, politologie a ostatní humanitní a společenskovědní obory}
\vfill


\vspace{20mm}

{\LARGE\bfseries \#filterbubble}

\vfill

\begin{tabular}{ll}
\bfseries Autoři: & Františka Sandroni\\
                  & Jakub Dostál\\
\noalign{\vspace{2mm}}
% \bfseries Konzultant: & Chceme nějakého?, UP Olomouc\\
%\bfseries Institute: & Department of Mathematical Analysis and Applications\\&  of Mathematics, Faculty of Science, Palack\'y University
\end{tabular}

\vfill
Olomouc 2017
\end{center}

% ##############################################################################
% ##############################################################################

\newpage
\openright

\noindent
\subsection*{Poděkování}
\noindent Rádi bychom poděkovali Dott. Kvidu Sandroni za projevené pozorné připomínky během celé tvorby naší práce, každá sebemenší myšlenka nám byla cennou radou a podnětem ke zdokonalování studie. Velký dík patří také Barboře Pomykalové za grafickou úpravu. Neméně děkujeme všem, kteří se ochotně podíleli na korektuře práce.


% ##############################################################################
% ##############################################################################

\newpage
%%% Page containing a legal statement
\vspace*{\stretch{8}}

\noindent
Prohlašujeme, že jsme tuto práci vypracovali samostatně a~výhradně
s~použitím citovaných pramenů, literatury a~dalších odborných zdrojů.

Prohlašujeme, že tištěná verze a elektronická verze soutěžní práce SOČ jsou
shodné.

\vspace{20mm}
\noindent
%% Place and date of signature
V~Olomouci dne \makebox[2.5cm]{\dotfill}
\hspace*{\fill}
\makebox[5cm]{\dotfill}
\hspace*{\fill}
\\
\\
\\
\makebox[5.55cm]{}
\hspace*{\fill}
\makebox[5cm]{\dotfill}
\hspace*{\fill}

\vspace*{\stretch{1}}

% ##############################################################################
% ##############################################################################

\newpage
%%% Czech and English abstracts
\nobreak\vbox to 0.49\vsize{
\setlength\parindent{0mm}
\setlength\parskip{5mm}
\newcommand{\forceindent}{\leavevmode{\parindent=1em\indent}}

\textbf{Title}: \#filterbubble\\
\textbf{Authors}: Františka Sandroni, Jakub Dost\'al\\
\textbf{School}: Slavonic grammar school Olomouc\\
% \textbf{Supervisor}:
% Tom\'aš F\"urst, Department of Mathematical Analysis and Applications of Mathematics, Faculty of Science, Palack\'y University Olomouc
\textbf{Abstract:}\\
Nowadays one of the mostly used source of the information are social networks. But the content user watches is adjusted by \textit{preferential algorithms}. They filter information and lead one to situation, when the content he wathes is nott fully balanced. The communities are constituted from the people with same initial opinion. This phenomena is called the \textit{filter bubble} and it causes \textbf{[[[can cause (zalezi na tom co chceme skutecne rict)]]]} creation of the extremism in the population, because it avoids users to see wide variety of opinions. And furthermore confirms the user that his opinion is the only opinion by shoving him lot of content with same orientation.\\
\forceindent We introduce the filter bubble and it's positives and negatives firstly. We present new methodology of filter bubble studying on the data from the \textit{Twitter} afterwards. The data are collected and then processed by the \textit{sentimental analysis}. In contrast to state of the art researche we are using with larger number of subjects directly from Twitter. At the end we show positive and negative features of the new methodology on few examples.\\
\textbf{[[[pokud budeme mluvit i o tom, ze trump ib nevytvari, musime pridat neco na konec abstraktu]]]}\\
\textbf{Keywords}: filterbubble, social media, sentiment analysis, democracy, society
\vss}


\vbox to 0.5\vsize{
\setlength\parindent{0mm}
\setlength\parskip{5mm}
\newcommand{\forceindent}{\leavevmode{\parindent=1em\indent}}

\textbf{Název práce}: \#filterbubble\\
\textbf{Autoři:} Františka Sandroni, Jakub Dost\'al\\
\textbf{Škola}: Slovanské gymnázium Olomouc\\
\textbf{Abstrakt:}\\
Významným zdrojem informací v dnešní době jsou sociální sítě. Obsah, který na nich jedinec pozoruje je však ovlivňověn \textit{preferenčními algoritmy}. Ty filtrují informace a vedou jedince do situace, kdy okruh příspěvků, které pozoruje, není plně vyvážený. Tím pádem se zde vytvářejí komunity, jež výrazně modifikují objem a obsah informací ovlivňující jejich členy. Tomuto fenoménu se říká \textit{filter bubble} a může vést například k samovolnému vzniku extrémistických názorů a skupin, neboť jedinci v dané bublině nemají přístup k dostatečně širokému spektru informací, ale pouze k zúženému výběru.\\
\forceindent Práce nejprve představuje \textit{informační bublinu} a její positiva a negativa. Poté prezentujeme novou metodiku výzkumu informační bubliny na datech poskytovaných \textit{Twitterem}. Tato data jsou nejprve pečlivě vybrána a následně zpracovávána pomocí \textit{sentimentální analýzy}. Narozdíl od mnoha předešlých studií pracujeme s větším počtem subjektů a jejich konkrétním chováním na sociální síti. Funkčnost našeho modelu ukážeme na několika jednoduchých příkladech, na nichž představíme jejich kladné i záporné vlastnosti.\\
\textbf{Klíčová slova}: informační bublina, sociální sítě, sentimentální analýza, demokracie, společnost
\vss}

% ##############################################################################
% ##############################################################################

\newpage
\openright

\pagestyle{plain}
\setcounter{page}{1}

\tableofcontents
