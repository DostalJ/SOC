\pagestyle{empty}
\begin{center}

\medskip

\vfill
{\bfseries\Large{Středoškolská odborná činnost}}

\medskip

{{\bfseries Obor: } 1. Matematika a statistika}
\vfill


\vspace{20mm}

% Title in English according to the official assignment
{\LARGE\bfseries The best title ever}

\vfill

\begin{tabular}{ll}
\bfseries Autor: & Jakub Dostál, Slovanské gymnázium Olomouc\\
\noalign{\vspace{2mm}}
\bfseries Konzultant: & Chceme nějakého?, UP Olomouc\\
%\bfseries Institute: & Department of Mathematical Analysis and Applications\\&  of Mathematics, Faculty of Science, Palack\'y University
\end{tabular}

\vfill
Olomouc 2016
\end{center}

%%%%%%%%%%%%%%%%%%%%%%%%%%%%%%%%%%%%%%%%%%%%%%%%
%%%%%%%%%%%%%%%%%%%%%%%%%%%%%%%%%%%%%%%%%%%%%%%%
%%%%%%%%%%%%%%%%%%%%%%%%%%%%%%%%%%%%%%%%%%%%%%%%

\newpage
\openright

\noindent
\subsection*{Poděkování}
\noindent Neděkujeme nikomu, jsme prostě nejlepší sami o sobě


%%%%%%%%%%%%%%%%%%%%%%%%%%%%%%%%%%%%%%%%%%%%%%%%
%%%%%%%%%%%%%%%%%%%%%%%%%%%%%%%%%%%%%%%%%%%%%%%%
%%%%%%%%%%%%%%%%%%%%%%%%%%%%%%%%%%%%%%%%%%%%%%%%

\newpage
%%% Page containing a legal statement
\vspace*{\stretch{8}}

\noindent
Prohlašuji, že jsem tuto práci vypracoval samostatně a~výhradně
s~použitím citovaných pramenů, literatury a~dalších odborných zdrojů.

Prohlašuji, že tištěná verze a elektronická verze soutěžní práce SOČ jsou
shodné.

\vspace{20mm}
\noindent
%% Place and date of signature
V~Olomouci dne \makebox[2.5cm]{\dotfill}
\hspace*{\fill}
\makebox[3cm]{\dotfill}
\hspace*{\fill}

\vspace*{\stretch{1}}

%%%%%%%%%%%%%%%%%%%%%%%%%%%%%%%%%%%%%%%%%%%%%%%%
%%%%%%%%%%%%%%%%%%%%%%%%%%%%%%%%%%%%%%%%%%%%%%%%
%%%%%%%%%%%%%%%%%%%%%%%%%%%%%%%%%%%%%%%%%%%%%%%%

\newpage
%%% Czech and English abstracts
\nobreak\vbox to 0.49\vsize{
\setlength\parindent{0mm}
\setlength\parskip{5mm}

Title:
New ways to simulate epidemic spreading in human society

Author:
Jakub Dost\'al

School:
Slavonic grammar school Olomouc

Supervisor:
Tom\'aš F\"urst, Department of Mathematical Analysis and Applications of Mathematics, Faculty of Science, Palack\'y University Olomouc

Abstract:\\
Society needs to fight against dangerous diseases. However, disease may stand for much more abstract process, for instance a fashion wave.

The goal of the work is to show the need to separate the topology of the network of human relations from the behaviour of the disease itself. This will reveal us substantive dependence of the simulation results on the structure of society.

First basic ways of modelling epidemic spreading are shown, both deterministic and stochastic. Traditional methods of simulation on complex networks are recalled. Then a new algorithm is presented separating behaviour of the disease from the features of the network itself.

Keywords:
epidemiology, complex networks, SI, SIS, SIR, SIRS
\vss}

%%%%%%%%%%%%%%%%%%%%%%%%%%%%%%%%%

\vbox to 0.5\vsize{
\setlength\parindent{0mm}
\setlength\parskip{5mm}

Název práce:
Nový způsob simulace šíření epidemie ve společnosti

Autor:
Jakub Dostál

Škola:
Slovanské gymnázium Olomouc

Konzultant:
Tomáš Fürst, Katedra matematické analýzy a aplikací matematiky, Přírodovědecká fakulta, Univerzita Palackého Olomouc

Abstrakt:\\
I~v~dnešní době společnost neustále pociťuje potřebu bojovat proti nebezpečným chorobám. Pod pojmem choroba si však můžeme představit i mnohem abs\-trakt\-něj\-ší pojmy, jako je třeba módní trend.

Cílem práce je ukázat potřebu oddělení vlastností topologie vztahů lidí ve spo\-leč\-nos\-ti, kde se epidemie šíří, a samotného charakteru choroby. Tím zároveň ukazujeme velkou závislost výpočtů právě na struktuře spo\-leč\-nos\-ti.

Nejprve uvádíme základní způsoby modelování, a to jak deterministické, tak stochastické. Ukazujeme tradiční způsoby simulací na komplexních sítích. Poté představujeme nový algoritmus oddělující vlastnosti sítě od vlastností samotné choroby.

Klíčová slova:
epidemiologie, komplexní sítě, SI, SIS, SIR, SIRS

\vss}


%%%%%%%%%%%%%%%%%%%%%%%%%%%%%%%%%%%%%%%%%%%%%%%%
%%%%%%%%%%%%%%%%%%%%%%%%%%%%%%%%%%%%%%%%%%%%%%%%
%%%%%%%%%%%%%%%%%%%%%%%%%%%%%%%%%%%%%%%%%%%%%%%%

\newpage
\openright

\pagestyle{plain}
\setcounter{page}{1}

\tableofcontents
