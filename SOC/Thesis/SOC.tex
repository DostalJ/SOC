%%%%% Single page layout:
\documentclass[12pt, a4paper]{article}
\usepackage[czech]{babel}
\usepackage[utf8]{inputenc}
\setlength\textwidth{145mm}
\setlength\textheight{247mm}
\setlength\oddsidemargin{15mm}
\setlength\evensidemargin{15mm}
\setlength\topmargin{0mm}
\setlength\headsep{0mm}
\setlength\headheight{0mm}
\let\openright=\clearpage

%%% Additional useful packages
\usepackage{amsmath}
\usepackage{amsfonts}
\usepackage{amsthm}
\usepackage{bm}
\usepackage{graphicx}
\usepackage{psfrag}
\usepackage{fancyvrb}
\usepackage{bbding}
\usepackage{dcolumn}
\usepackage{booktabs}
\usepackage{paralist}
\usepackage{indentfirst}
\usepackage[nottoc]{tocbibind}
\usepackage[unicode]{hyperref}
\usepackage{subfig} 	%subfloats.

%%% \subref*
\captionsetup[subfigure]{subrefformat=simple,labelformat=simple,listofformat=subsimple}
\renewcommand\thesubfigure{(\alph{subfigure})}

\numberwithin{equation}{section} 	%cislovani rovnic pro article
\interfootnotelinepenalty=10000	%aby se \footnote nerozdeloval na dve strany


% ####################################################################
% ####################################################################
\begin{document}
\pagestyle{empty}
\begin{center}

\medskip

\vfill
{\bfseries\Large{Středoškolská odborná činnost}}

\medskip

{{\bfseries Obor: } 17. Filozofie, politologie a ostatní humanitní a společenskovědní obory}
\vfill


\vspace{20mm}

{\LARGE\bfseries \#filterbubble}

\vfill

\begin{tabular}{ll}
\bfseries Autoři: & Francesca Sandroni,\hspace{0.153cm} Slovanské gymnázium Olomouc\\
                  & Jakub Dostál, Slovanské gymnázium Olomouc\\
\noalign{\vspace{2mm}}
% \bfseries Konzultant: & Chceme nějakého?, UP Olomouc\\
%\bfseries Institute: & Department of Mathematical Analysis and Applications\\&  of Mathematics, Faculty of Science, Palack\'y University
\end{tabular}

\vfill
Olomouc 2016
\end{center}

% ##############################################################################
% ##############################################################################

\newpage
\openright

\noindent
\subsection*{Poděkování}
\noindent Neděkujeme nikomu, jsme prostě nejlepší sami o sobě


% ##############################################################################
% ##############################################################################

\newpage
%%% Page containing a legal statement
\vspace*{\stretch{8}}

\noindent
Prohlašujeme, že jsme tuto práci vypracovali samostatně a~výhradně
s~použitím citovaných pramenů, literatury a~dalších odborných zdrojů.

Prohlašujeme, že tištěná verze a elektronická verze soutěžní práce SOČ jsou
shodné.

\vspace{20mm}
\noindent
%% Place and date of signature
V~Olomouci dne \makebox[2.5cm]{\dotfill}
\hspace*{\fill}
\makebox[3cm]{\dotfill}
\hspace*{\fill}
\\
\\
\\
\makebox[5.55cm]{}
\hspace*{\fill}
\makebox[3cm]{\dotfill}
\hspace*{\fill}

\vspace*{\stretch{1}}

% ##############################################################################
% ##############################################################################

\newpage
%%% Czech and English abstracts
\nobreak\vbox to 0.49\vsize{
\setlength\parindent{0mm}
\setlength\parskip{5mm}

Title:
New ways to simulate epidemic spreading in human society

Author:
Jakub Dost\'al

School:
Slavonic grammar school Olomouc

Supervisor:
Tom\'aš F\"urst, Department of Mathematical Analysis and Applications of Mathematics, Faculty of Science, Palack\'y University Olomouc

Abstract:\\
Society needs to fight against dangerous diseases. However, disease may stand for much more abstract process, for instance a fashion wave.

The goal of the work is to show the need to separate the topology of the network of human relations from the behaviour of the disease itself. This will reveal us substantive dependence of the simulation results on the structure of society.

First basic ways of modelling epidemic spreading are shown, both deterministic and stochastic. Traditional methods of simulation on complex networks are recalled. Then a new algorithm is presented separating behaviour of the disease from the features of the network itself.

Keywords:
epidemiology, complex networks, SI, SIS, SIR, SIRS
\vss}


\vbox to 0.5\vsize{
\setlength\parindent{0mm}
\setlength\parskip{5mm}

Název práce:
Nový způsob simulace šíření epidemie ve společnosti

Autor:
Jakub Dostál

Škola:
Slovanské gymnázium Olomouc

Konzultant:
Tomáš Fürst, Katedra matematické analýzy a aplikací matematiky, Přírodovědecká fakulta, Univerzita Palackého Olomouc

Abstrakt:\\
I~v~dnešní době společnost neustále pociťuje potřebu bojovat proti nebezpečným chorobám. Pod pojmem choroba si však můžeme představit i mnohem abs\-trakt\-něj\-ší pojmy, jako je třeba módní trend.

Cílem práce je ukázat potřebu oddělení vlastností topologie vztahů lidí ve spo\-leč\-nos\-ti, kde se epidemie šíří, a samotného charakteru choroby. Tím zároveň ukazujeme velkou závislost výpočtů právě na struktuře spo\-leč\-nos\-ti.

Nejprve uvádíme základní způsoby modelování, a to jak deterministické, tak stochastické. Ukazujeme tradiční způsoby simulací na komplexních sítích. Poté představujeme nový algoritmus oddělující vlastnosti sítě od vlastností samotné choroby.

Klíčová slova:
epidemiologie, komplexní sítě, SI, SIS, SIR, SIRS

\vss}

% ##############################################################################
% ##############################################################################

\newpage
\openright

\pagestyle{plain}
\setcounter{page}{1}

\tableofcontents

\newpage
\section*{Úvod}
\noindent

% ##############################################################################
% ##############################################################################
\newpage
\section{Co před námi internet skrývá}
\noindent Ve světě plném technického pokroku a masového používání internetu se snadno můžeme ztrácet v množství přijímaných informací. Nepřeberné množství zdrojů nás dennodenně zahlcuje spoustou nových zpráv ať už na sociálních sítích, či kdekoliv jinde. Je složité se v nich orientovat, natož tak hledat a nalézat relevantní souvislosti, neboť nikde není psáno, které zprávy považovat za pravdivé a které naopak za nepravdivé zprávy anonymních komentátorů.

Role, tedy působnost žurnalistiky samotné se razantně mění. Dnešní novinář nechodí s poznámkovým blokem a tužkou v kapse, získané informace a souvislosti nesepisuje doma do celistvých článků poutajících pozornost čtenářů svou kvalitou, nýbrž se smartphonem v ruce fotí, natáčí a dělá živé reportáže přímo z místa.

Toto online vysílání a sdílení má kromě spousty výhod, jakožto rychlé informovanosti uživatelů od dění v silničním provozu, po nejaktuálnější výsledky voleb, také svá úskalí. Představme si kupříkladu nebezpečí atentátu, či atentát již spáchaný. Sociální sítě se začnou naplňovat spoustou příspěvků. Odhady počtu obětí, identita pachatelů a možnost dalšího ohrožení jsou hlavními tématy. Při\-spí\-vat do společné online diskuze smí každý, ať už kvalifikovaný publicista nebo náhodný autor. Zájem davů se upíná na dechberoucí události a čerstvé zprávy, proto každý příspěvek, zejména z místa činu, bude dále sdílen a zmiňován jako důvěryhodný. Kde je však záruka pravdivosti příspěvku? Nemůže nakonec dojít k situaci, kdy ten nejzajímavější, leč nepodložený fakty, bude brán ve vážnost u mas lidí právě více než ten fakty podložený, jen ne tak senzační?

Mají stále žurnalisté moc ovlivnit čtenáře a oslovit je svými pracemi, či je čtenářova pozornost upínána na senzace vyskytujíc se na zdech sociálních sítí? Facebook a především Twitter, fungující na rychlém tweetování zpráv, se dostává do popředí sdělovacích prostředků. Na jejich zdech sledujeme mnohé odkazy na komentáře, videa a články, formující se názory jedinců i davů, vznikající i zanikající politické myšlenky. Nakolik jsou tato samozvaná masmédia relevantní a kde mají své nedostatky, obzvláště v rovnoměrném šíření informací napříč společností, se budeme zabývat dále.

\subsection{Filter bubble}
\noindent \textit{Filter bubble} nebo také \textit{informační bublina} je jeden z mnoha fenoménů dnešní doby. Jako první na ni upozornil a popsal ji Eli Pariser~\cite{Pariser2011, PariserTed}. Jde o jev vyskytující se na sociálních sítích, kdy uživatel každou online aktivitou\footnote{Tím rozumějmě kliknutí, sdílení, komentování a obdobné činnosti.} poukazuje na oblasti jeho zájmů. Tyto algoritmy snažící se usnadnit život pomocí personalizace viděných informací se každou další činností uživatele zdokonalují, což v konečném výsledku znamená, že jsou to právě ony, jež rozhdují, co bude pro uživatlele viditelné a co naopak konsekvencí prací algoritmů bude uživateli podáno v menší míře, či úplně skryto v celkovém proudu informací~\cite{TheImpactOfFilterBubble}. Celkově to tedy může vést jedince do situace, kdy místo širokého spektra příspěvků na sobě obsahově nezávislých, vidí příspěvky jen takové, jež byly vybrány preferenčními algoritmy na základě jeho předešlé činnosti a tedy velmi zúžené škále informací.

Jak se ukázalo v předešlé studii~\cite{TheImpactOfFilterBubble}, filter bubble je velmi individuální a její efekt není u všech uživatelů zcela totožný. Míra filter bubble nezávisí pouze na obsahu viděných příspěvků, jak by se mohlo předpokládat, avšak na zdroji odkud informace čerpá. V důsledku můžeme pozorovat značně silnější efekt informační bubliny u uživatelů s mnoha konexemi na jiné uživatele, než u těch s menším rozsahem jejich spojení. Vycházíme-li z běžné praxe reálného života/prostředí, všímáme si, že jedinci s mnoha známostmi mají silnější postavení ve společnosti a tudíž i notný vliv na ostatní členy dané společnosti. Stejně tak je tomu na sociálních sítích. V situaci, kdy uživatel s mnoha konexemi na ostatní projeví svůj názor příspěvkem, či komentářem, je tu daleko větší pravděpodobnost ovlivnění značného množství uživatelů s ním spojených.

Informační bublina je velmi rozšířeným problémem, pokud ji si je však uživatel vědom, není filter bubble nepřekonatelnou bariérou v získávání relevantních informací.


\cite{wikiTwitter}


\subsection{Problémy filter bubble}
Vezmeme-li v úvahu, kolik času lidé tráví na sociálních sítích, je zřejmé, že jejich názory a postoje se primárně vytvářejí zrovna zde~\cite{TheImpactOfFilterBubble, BeyondFilterBubble, whyNewsOnTwitter}. Sleduje-li uživatel pouze názorově shodné příspěvky, mohla by informační bublina představovat značnou hrozbu demokratickým systémům, neboť uživatelům předkládá již vyfiltrované příspěvky a to zejména takové, jež by podpořili názor uživatele samotného, nikoli názor odlišný.

V jedné z předešlých studií~\cite{BeyondFilterBubble}, byla provedena řada experimentů odhalujících několik zajímavých efektů informační bubliny. Jako mnoho podobných studií však své výsledky vyvozuje z reakcí malého počtu lidí\footnote{V konkrétním případě této práce okolo 30 jedinců.} v uměle vytvořených situacích. Ukazují například, že přijímání názorů z různých úhlů pohledů závisí především na hloubce zájmu o dané téma. Čím vyšší je zájem o téma, tím vyšší je ochota přijímat protiargumenty, a naopak, čím nižší je zájem o téma, tím nižší je ochota přijímat protiargumenty.

V případě, kdy jedinec s nízkým povědomím o daném tématu se dostane ke zdroji informací a není motivován hlubším podnětem, jako jsou například blížící se volby, nevykazuje zájem o hledání relevantních faktů, nýbrž dává přednost vyhledávání \textit{users opinion}, nehledě na zdroje, o které se \textit{users opinion} opírá. Předložíme-li tedy současně jedinci informace podobné jeho již dříve získaným postojům a informace lišící se od jeho postoje, ve většině případů si vybere informace podobající se jeho stanovisku, avšak dostane-li se jedinec s nízkým povědomím do kritické situace\footnote{Kupříkladu je-li jedinec postižen nějakou nemocí je mnohem více motivován vyhledávat informace.} a je motivován se v daném tématu vzdělávat, začíná vyhledávat informace podložené fakty, ať už podporující jeho stanovisko, či nikoliv.

Takovéto chování stále se opakujícího výběru již známého obsahu může jedince vést do \textit{echo chamber}, kde tímto selektivním ziskem podobně motivovaných informací se utvrzuje ve svém původním názoru a naopak informace odlišné, či opačné vytěsňuje do takové míry, že není schopen jejich dalšího vnímání.

Informační bublina může zapříčinit nemalé následky v makroskopickém měřítku na celou dnešní společnost v mnoha ohledech. V první řadě upozorněme na zjevné riziko, které se naskytuje při aktivních preferenčních algoritmech mezi uživateli sociálních sítí v demokratických společnostech. Zde i přes nabízenou diversitu obsahu uživatel opět vidí jen omezenou část. Autoři v~\cite{BreakingTheFilterBubble} vychází z konceptu, kde demokracie jako taková je rozdělena na liberální, deliberativní, republikánskou a agonistickou a pozorují, že každá z nich je ohrožena v jiné části její struktury. Problémy způsobené informační bublinou v liberální demokracii, jakožto ztráta povědomí občanů o různorodosti volby a nezávislosti médií, coby primárního zdroje informací občanů zvrhle upadajícího do rukou úzkého okruhu lidí, a demokracie deliberativní, kde sledujeme nedostatky v rovnocenné občanské diskusi, klesající toleranci vůči odlišným názorům a úbytek obecného přání zisku nových epistemických argumentů, částečně řeší již popsané aplikace jako třeba \textit{Balancer}, \textit{Scoopion}, \textit{ConsiderIt}, \textit{Opinion space} a další. Žádná však neřeší ohrožení v typech republikánské a agonistické demokracie.

Otázkou proto zůstává, jak dostatečně rozpoznat míru filter bubble a ochránit rozhled uživatele sociálních sítí bez ohledu na charakter demokracie, ve které se vyskytuje. Představme si živé předvolební období, kdy politické strany vytáhnou do boje a nebojí se použít žádných prostředků k potupě politických rivalů, kdy jedna aféra stíhá druhou, a na povrch vyplouvají rozličné skandály představitelů politických stran. Zároveň jsou také vypouštěny různé výstižné slogany rádoby řešící lokální i globální problémy. Čím více zaujatý slogan, tím masovější ovace. Začíná davové šílenství v podobě obrovských internetových diskusí v tématech, jež jsou pro uživatele klíčová. Vyhledáváním a připojováním se ke společenství se jedinec cítí být více informovaný, nicméně ztrácí přehled o celém tématu a zaměřuje se na čím dál menší okruh informací podporující jeho názor. Kamkoli se podívá a cokoli i přečte je uspokojen, vidí stále příspěvky podobné jeho názorům. Jak je dobře známo, uživatelé těchto internetových diskusí jsou často svým přesvědčením uchváceni natolik, že ztrácí veškeré zábrany racionálně smýšlející osoby a své zaujaté názory se nebojí ukazovat široké veřejnosti~\cite{DemocracyOnline}. Je-li však náhled na téma již ze začátku extremistický, kam až může zajít? Co když tyto podporované politické strany proklamující se všeobecnými předsudky nejsou vhodnou volbou pro stát, ale díky svým hojně sdíleným příspěvkům oslovují více a více lidí, kteří dále šíří ideologii? Informační bubliny se z tohoto hlediska stávají problémem, i co se týče jejich etického vlivu na společnost/vnímání jejich etiky.


\subsection{Výhody filter bubble}
\subsection{Preferenční algoritmy}
\noindent Jak jsme již výše zmínili, personalizace obsahum který vidíme na internetu může být jak velkým problémem, tak velkou výhodou. Čemu jsme však dosud nevěnovali pozornost je jak informační bubliny vznikají. Je zřejmé, že \textit{Facebook}, \textit{Twitter}, \textit{YouTube} a podobné internetové giganty shromažďují velké množství dat o našich internetových aktivitách. Méně jasné je, že to nedělají kvůli zlomyslným plánům na ovládutí světa, nýbrž kvůli snaze zpříjemnit užívání jejich služeb\footnote{Tím, že nám usnadní a zpříjemní jejich užívání si zajistí větší návštěvnost, což je zdrojem jejich příjmů.}. Tato data poté pomocí moderních matematických a statistických metod užívají například k výběru obsahu, který nám bude co nejvíce imponovat, respektive k výberu obsahu, o kterém jsme se již dříve výjádřili, že je pro nás zajímavý.

S obdobným přístupem se můžeme setkat při online nakupování~\cite{Amazon}, kde jsou nám doporučovány produkty obdobné těm, které jsme v poslední době hledali. Stejně tak například na \textit{YouTube}~\cite{YouTube}, kde se dříve dostaneme k videím s podobným obsahem, jaký často sledujeme.

Metody, které se užívají pro sociální sítě a informační kanály~\cite{TwitterRecomendation} jsou většinou velmi sofistikované a opírají se o hluboké znalosti \textit{machine learningu}, \textit{statistiky} a \textit{data mining}. Velmi zjednodušeně řečeno, nový uživatel určité stránky je nejprve vystaven velmi širokému spektru informací. Někde na serveru provozovatele sítě sedí malá ne příliš chytrá umělá inteligence, která si zapisuje na co uživatel kliká\footnote{Samozřejmě také sleduje další způsoby hodnocení příspěvků, které jsou pro různé stránky odlišné. Na \textit{Facebooku} například \textit{like} na \textit{Twitteru} \textit{retweet}}. Své zápisky následně zpracovává a pomocí těchto pracovaných poznámek následně odhaduje co by se danému uživateli mohlo líbit.






% ##############################################################################
% ##############################################################################
\newpage
\section{Metodika sběru dat}
\noindent S pokrokem v oblasti technologie probíhá i obrovský pokrok v oblasti čerpání zpráv a informací. Narozdíl od minulosti, drtivá většina obyvatel vyspělých států má přístup k novinkám v průběhu celého dne. Denní počet příspěvků na \textit{Twitteru} dosahoval v roce 2010 neskutečných 6 milionů~\cite{Mathioudakis2010}. Je zřejmé, že není možné používat stejné postupy pro analýzu takto rozsáhlého množství dat jako dříve. Naštěstí se v již nějakou dobu významným tempem posouvá i oblast \textit{Big Data, Machine Learningu} a jejich podoblastí. Tyto obory nacházejí uplatnění téměř ve všech oblastech dnešní vědy~\cite{Huberman2012-2-15} a stále větší pozornosti se jim dostává i v sociologii~\cite{Tinati2014, McFarland2016, Shah2015-04-09}.

\subsection{Data z Twitteru}
\noindent Právě data z Twitteru jsou pro nás velmi vhodná. Jak se ukazuje~\cite{whyNotFb}, narozdíl on \textit{Facebooku}, Twitter užívá mnoho lidí jako zdroj informací a zpráv o aktuálním dění ve světě. Samozřejmě můžeme zpochybňovat validitu a přesnost informací, které se na takovýchto sítích objěvují. Ať už jsou takovéto obavy oprávněné, nebo ne, vůzkum sociologických jevů, který v této práci provádíme to nijak neovlivnuje.

Dalším důležitým důvodem, který vedl k výběru Twitteru jako média, ze kterého budeme stahovat informace je snadná přístupnost k datům pomocí služby API poskytovaný právě přímo Twitterem~\cite{twitterAPI}. Konkrétně pro naše účely jsme tuto službu nepoužívali přímo, ale za pomocí balíčku \textit{tweepy}~\cite{tweepy} pro programovací jazyk \textit{python}. Ten umožňuje velmi snadné ovládání a filtrování proudu dat, které si vyžádáme Twitteru a také jejich okamžitou analýzu a zpracování v pythonu.

\subsection{Sentimentální analýza textu}
\noindent


% ##############################################################################
% ##############################################################################
\newpage
\section{Konstrukce měření}
\subsection{Výběr pozorovaných vzorků}
\subsection{Metodika výběru vzorků}





\newpage
\section{Závěr}
\noindent

\newpage
\begin{thebibliography}{99}

    \bibitem{DemocracyOnline}
    PAPACHARISSI, Zizi. Democracy online: civility, politeness, and the democratic potential of online political discussion groups. \textit{New Media}. 2004, \textbf{6}(2), 259-283. DOI: 10.1177/1461444804041444. ISSN 1461-4448. Dostupné také z: \url{http://journals.sagepub.com/doi/10.1177/1461444804041444}

    \bibitem{BreakingTheFilterBubble}
    BOZDAG, Engin a Jeroen VAN DEN HOVEN. Breaking the filter bubble: democracy and design. \textit{Ethics and Information Technology}. 2015, \textbf{17}(4), 249-265. DOI: 10.1007/s10676-015-9380-y. ISSN 1388-1957. Dostupné také z: \url{http://link.springer.com/10.1007/s10676-015-9380-y}

    \bibitem{TwitterRecomendation}
    KIM, Younghoon a Kyuseok SHIM. TWITOBI: A Recommendation System for Twitter Using Probabilistic Modeling. \textit{2011 IEEE 11th International Conference on Data Mining}. IEEE, 2011, , 340-349. DOI: 10.1109/ICDM.2011.150. ISBN 978-1-4577-2075-8. Dostupné také z: \url{http://ieeexplore.ieee.org/document/6137238/}

    \bibitem{YouTube}
    DAVIDSON, James, Blake LIVINGSTON, Dasarathi SAMPATH, et al. The YouTube video recommendation system. \textit{Proceedings of the fourth ACM conference on Recommender systems - RecSys '10}. New York, New York, USA: ACM Press, 2010, , 293-. DOI: 10.1145/1864708.1864770. ISBN 9781605589060. Dostupné také z: \url{http://portal.acm.org/citation.cfm?doid=1864708.1864770}

    \bibitem{Amazon}
    LINDEN, G., B. SMITH a J. YORK. Amazon.com recommendations: item-to-item collaborative filtering. \textit{IEEE Internet Computing}. 2003, \textbf{7}(1), 76-80. DOI: 10.1109/MIC.2003.1167344. ISSN 1089-7801. Dostupné také z: \url{http://ieeexplore.ieee.org/document/1167344/}

    \bibitem{BeyondFilterBubble}
    LIAO, Q. Vera a Wai-Tat FU. Beyond the filter bubble: Interactive effects of perceived threat and topic involvement on selective exposure to information. \textit{Proceedings of the SIGCHI Conference on Human Factors in Computing Systems - CHI '13}. New York, New York, USA: ACM Press, 2013, 2359-2368. DOI: 10.1145/2470654.2481326. ISBN 9781450318990. Dostupné také z: \url{http://dl.acm.org/citation.cfm?doid=2470654.2481326}

    \bibitem{TheImpactOfFilterBubble}
    GOTTRON, Thomas a Felix SCHWAGEREIT. The Impact of the Filter Bubble -- A Simulation Based Framework for Measuring Personalisation Macro Effects in Online Communities. \textit{ARXIV: Computer Science - Social and Information Networks}. 2016, \textbf{2016}. 2016arXiv161206551G. Dostupné také z: \url{https://arxiv.org/abs/1612.06551\#}

    \bibitem{PariserTed}
    PARISER, Eli. (2011). \textit{Beware online "filter bubbles"} [online]. Dostupné z \url{https://www.ted.com/talks/eli_pariser_beware_online_filter_bubbles}

    \bibitem{Pariser2011}
    PARISER, Eli. \textit{The filter bubble: what the Internet is hiding from you}. New York: Penguin Press, 2011. ISBN 15-942-0300-8.

    \bibitem{wikiTwitter}
    Twitter. In: \textit{Wikipedia: the free encyclopedia} [online]. San Francisco (CA): Wikimedia Foundation, 2017, 2017-01-22 [cit. 2017-01-22]. Dostupné z: \url{https://en.wikipedia.org/wiki/Twitter}

    \bibitem{whyNotFb}
    BARTHEL, MICHAEL, ELISA SHEARER, JEFFREY GOTTFRIED a AMY MITCHELL. \textit{The Evolving Role of News on Twitter and Facebook} [online]. In: . Pew Research Center, 2015 [cit. 2017-01-22]. Dostupné z: \url{http://www.journalism.org/2015/07/14/the-evolving-role-of-news-on-twitter-and-facebook/}

    \bibitem{whyNewsOnTwitter}
    Twitter and the News: How people use the social network to learn about the world. \textit{American Press Institute: Insights, tools and research to advance journalism} [online]. 2015, \textbf{2015} [cit. 2017-01-22]. Dostupné z: \url{https://www.americanpressinstitute.org/publications/reports/survey-research/how-people-use-twitter-news/}

    \bibitem{Mathioudakis2010}
    MATHIOUDAKIS, Michael a Nick KOUDAS. TwitterMonitor. \textit{Proceedings of the 2010 international conference on Management of data - SIGMOD '10}. New York, New York, USA: ACM Press, 2010, \textbf{2010}, 1155-1158. DOI: 10.1145/1807167.1807306. ISBN 9781450300322. Dostupné také z: \url{http://portal.acm.org/citation.cfm?doid=1807167.1807306}

    \bibitem{Tinati2014} % big data in sociology
    TINATI, Ramine, Susan HALFORD, Leslie CARR a Catherine POPE. Big Data: Methodological Challenges and Approaches for Sociological Analysis. \textit{Sociology}. 2014, \textbf{48}(4), 663-681. DOI: 10.1177/0038038513511561. ISSN 0038-0385. Dostupné také z: \url{http://journals.sagepub.com/doi/10.1177/0038038513511561}

    \bibitem{McFarland2016} % big data in sociology
    MCFARLAND, Daniel A., Kevin LEWIS a Amir GOLDBERG. Sociology in the Era of Big Data: The Ascent of Forensic Social Science. \textit{The American Sociologist}. 2016, \textbf{47}(1), 12-35. DOI: 10.1007/s12108-015-9291-8. ISSN 0003-1232. Dostupné také z: \url{http://link.springer.com/10.1007/s12108-015-9291-8}

    \bibitem{Shah2015-04-09} % big data in sociology
    SHAH, D. V., J. N. CAPPELLA a W. R. NEUMAN. Big Data, Digital Media, and Computational Social Science: Possibilities and Perils. \textit{The ANNALS of the American Academy of Political and Social Science}. 2015, \textbf{659}(1), 6-13. DOI: 10.1177/0002716215572084. ISSN 0002-7162. Dostupné také z: \url{http://ann.sagepub.com/cgi/doi/10.1177/0002716215572084}

    \bibitem{Huberman2012-2-15} % big data anywhere
    HUBERMAN, Bernardo A. Sociology of science: Big data deserve a bigger audience. \textit{Nature}. 2012-2-15, \textbf{482}(7385), 308-308. DOI: 10.1038/482308d. ISSN 0028-0836. Dostupné také z: \url{http://www.nature.com/doifinder/10.1038/482308d}

    \bibitem{tweepy}
    HILL, Aaron a Joshua ROESSLEIN. Tweepy. \textit{Github} [online]. [cit. 2017-01-22]. Dostupné z: \url{https://github.com/tweepy/tweepy}

    \bibitem{twitterAPI}
    API Overview. \textit{Twitter Developer Documentation} [online]. Twitter, 2016 [cit. 2017-01-22]. Dostupné z: \url{https://dev.twitter.com/overview/api}
\end{thebibliography}
% ZKONTROLUJ \TEXTBF V CITACICH

% ##############################################################################
% ##############################################################################
\newpage
\section*{Přílohy}
\addcontentsline{toc}{section}{Přílohy}
\subsection*{Prvni kapitola přiloh}
\noindent
\end{document}
