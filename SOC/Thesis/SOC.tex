%%%%% Single page layout:
\documentclass[12pt, a4paper]{article}
\usepackage[czech]{babel}
\usepackage[utf8]{inputenc}
\setlength\textwidth{145mm}
\setlength\textheight{247mm}
\setlength\oddsidemargin{15mm}
\setlength\evensidemargin{15mm}
\setlength\topmargin{0mm}
\setlength\headsep{0mm}
\setlength\headheight{0mm}
\let\openright=\clearpage

%%% Additional useful packages
\usepackage{amsmath}
\usepackage{amsfonts}
\usepackage{amsthm}
\usepackage{bm}
\usepackage{graphicx}
\usepackage{psfrag}
\usepackage{fancyvrb}
\usepackage{bbding}
\usepackage{dcolumn}
\usepackage{booktabs}
\usepackage{paralist}
\usepackage{indentfirst}
\usepackage[nottoc]{tocbibind}
\usepackage[unicode]{hyperref}
\usepackage{subfig} 	%subfloats.

%%% \subref*
\captionsetup[subfigure]{subrefformat=simple,labelformat=simple,listofformat=subsimple}
\renewcommand\thesubfigure{(\alph{subfigure})}
%%%
\numberwithin{equation}{section} 	%cislovani rovnic pro article
\usepackage{physics}	%nice, standing differentials
\usepackage[]{units} %pěkné jednotky
\interfootnotelinepenalty=10000	%aby se \footnote nerozdeloval na dve strany


% ####################################################################
% ####################################################################
% ####################################################################
% ####################################################################


\begin{document}
\pagestyle{empty}
\begin{center}

\medskip

\vfill
{\bfseries\Large{Středoškolská odborná činnost}}

\medskip

{{\bfseries Obor: } 17. Filozofie, politologie a ostatní humanitní a společenskovědní obory}
\vfill


\vspace{20mm}

{\LARGE\bfseries \#filterbubble}

\vfill

\begin{tabular}{ll}
\bfseries Autoři: & Františka Sandroni\\
                  & Jakub Dostál\\
\noalign{\vspace{2mm}}
% \bfseries Konzultant: & Chceme nějakého?, UP Olomouc\\
%\bfseries Institute: & Department of Mathematical Analysis and Applications\\&  of Mathematics, Faculty of Science, Palack\'y University
\end{tabular}

\vfill
Olomouc 2017
\end{center}

% ##############################################################################
% ##############################################################################

\newpage
\openright

\noindent
\subsection*{Poděkování}
\noindent Rádi bychom poděkovali Dott. Kvidu Sandroni za projevené pozorné připomínky během celé tvorby naší práce, každá sebemenší myšlenka nám byla cennou radou a podnětem ke zdokonalování studie. Velký dík patří také Barboře Pomykalové za grafickou úpravu. Neméně děkujeme všem, kteří se ochotně podíleli na korektuře práce.


% ##############################################################################
% ##############################################################################

\newpage
% Page containing a legal statement
\vspace*{\stretch{8}}

\noindent
Prohlašujeme, že jsme tuto práci vypracovali samostatně a~výhradně
s~použitím citovaných pramenů, literatury a~dalších odborných zdrojů.

Prohlašujeme, že tištěná verze a elektronická verze soutěžní práce SOČ jsou
shodné.

\vspace{20mm}
\noindent
%% Place and date of signature
V~Olomouci dne \makebox[2.5cm]{\dotfill}
\hspace*{\fill}
\makebox[5cm]{\dotfill}
\hspace*{\fill}
\\
\\
\\
\makebox[5.55cm]{}
\hspace*{\fill}
\makebox[5cm]{\dotfill}
\hspace*{\fill}

\vspace*{\stretch{1}}

% ##############################################################################
% ##############################################################################

\newpage
%%% Czech and English abstracts
\nobreak\vbox to 0.49\vsize{
\setlength\parindent{0mm}
\setlength\parskip{5mm}
\newcommand{\forceindent}{\leavevmode{\parindent=1em\indent}}

\textbf{Title}: \#filterbubble\\
\textbf{Authors}: Františka Sandroni, Jakub Dost\'al\\
\textbf{School}: Slavonic grammar school Olomouc\\
% \textbf{Supervisor}:
% Tom\'aš F\"urst, Department of Mathematical Analysis and Applications of Mathematics, Faculty of Science, Palack\'y University Olomouc
\textbf{Abstract:}\\
Nowadays one of the mostly used source of the information are social networks. But the content user watches is adjusted by \textit{preferential algorithms}. They filter information and lead one to situation, when the content he wathes is nott fully balanced. The communities are constituted from the people with same initial opinion. This phenomena is called the \textit{filter bubble} and it causes \textbf{[[[can cause (zalezi na tom co chceme skutecne rict)]]]} creation of the extremism in the population, because it avoids users to see wide variety of opinions. And furthermore confirms the user that his opinion is the only opinion by shoving him lot of content with same orientation.\\
\forceindent We introduce the filter bubble and it's positives and negatives firstly. We present new methodology of filter bubble studying on the data from the \textit{Twitter} afterwards. The data are collected and then processed by the \textit{sentimental analysis}. In contrast to state of the art researche we are using with larger number of subjects directly from Twitter. At the end we show positive and negative features of the new methodology on few examples.\\
\textbf{[[[pokud budeme mluvit i o tom, ze trump ib nevytvari, musime pridat neco na konec abstraktu]]]}\\
\textbf{Keywords}: filterbubble, social media, sentiment analysis, democracy, society
\vss}


\vbox to 0.5\vsize{
\setlength\parindent{0mm}
\setlength\parskip{5mm}
\newcommand{\forceindent}{\leavevmode{\parindent=1em\indent}}

\textbf{Název práce}: \#filterbubble\\
\textbf{Autoři:} Františka Sandroni, Jakub Dost\'al\\
\textbf{Škola}: Slovanské gymnázium Olomouc\\
\textbf{Abstrakt:}\\
Významným zdrojem informací v dnešní době jsou sociální sítě. Obsah, který na nich jedinec pozoruje je však ovlivěn \textit{preferenčními algoritmy}. Ty filtrují informace a vedou jedince do situace, kdy okruh příspěvků, které pozoruje, není plně vyvážený. Tím pádem se zde vytvářejí komunity, jež výrazně modifikují objem a obsah informací ovlivňující jejich členy. Tomuto fenoménu se říká \textit{filter bubble} a může vést například k samovolnému vzniku extrémistických názorů a skupin, neboť jedinci v dané bublině nemají přístup k dostatečně širokému spektru informací, ale pouze k zúženému výběru.\\
\forceindent Práce nejprve představuje \textit{informační bublinu} a její positiva a negativa. Poté prezentujeme novou metodiku výzkumu informační bubliny na datech poskytovaných \textit{Twitterem}. Tato data jsou nejprve pečlivě vybrána a následně zpracována pomocí \textit{sentimentální analýzy}. Narozdíl od mnoha předešlých studií pracujeme s větším počtem subjektů a jejich konkrétním chováním na sociální síti. Funkčnost našeho modelu ukážeme na několika jednoduchých příkladech, na nichž představíme jejich kladné i záporné vlastnosti.\\
\textbf{Klíčová slova}: informační bublina, sociální sítě, sentimentální analýza, demokracie, společnost
\vss}

% ##############################################################################
% ##############################################################################

\newpage
\openright

\pagestyle{plain}
\setcounter{page}{1}

\tableofcontents

\newpage
\section*{Úvod}
\noindent


\newpage
\section{Co je to infromační bublina}
\noindent Ve světě plném technického pokroku a masového používání internetu se snadno můžeme ztrácet v množství přijímaných informací. Nepřeberné množství zdrojů nás dennodenně zahlcuje spoustou nových zpráv ať už na sociálních sítích, či kdekoliv jinde. Je složité se v nich orientovat, natož tak hledat a nalézat relevantní souvislosti, neboť nikde není psáno, které zprávy považovat za pravdivé a které naopak za nepravdivé zprávy anonymních komentátorů.

Role, tedy působnost žurnalistiky samotné se razantně mění. Dnešní novinář nechodí s poznámkovým blokem a tužkou v kapse, získané informace a souvislosti nesepisuje doma do celistvých článků poutajících pozornost čtenářů svou kvalitou, nýbrž se smartphonem v ruce fotí, natáčí a dělá živé reportáže přímo z místa.

Toto online vysílání a sdílení má kromě spousty výhod, jakožto rychlé informovanosti uživatelů od dění v silničním provozu, po nejaktuálnější výsledky voleb, také svá úskalí. Představme si kupříkladu nebezpečí atentátu, či atentát již spáchaný. Sociální sítě se začnou naplňovat spoustou příspěvků. Odhady počtu obětí, identita pachatelů a možnost dalšího ohrožení jsou hlavními tématy. Při\-spí\-vat do společné online diskuze smí každý, ať už kvalifikovaný publicista nebo náhodný autor. Zájem davů se upíná na dechberoucí události a čerstvé zprávy, proto každý příspěvek, zejména z místa činu, bude dále sdílen a zmiňován jako důvěryhodný. Kde je však záruka pravdivosti příspěvku? Nemůže nakonec dojít k situaci, kdy ten nejzajímavější, leč nepodložený fakty, bude brán ve vážnost u mas lidí právě více než ten fakty podložený, jen ne tak senzační?

Mají stále žurnalisté moc ovlivnit čtenáře a oslovit je svými pracemi, či je čtenářova pozornost upínána na senzace vyskytujíc se na zdech sociálních sítí? Facebook a především Twitter, fungující na rychlém tweetování zpráv, se dostává do popředí sdělovacích prostředků. Na jejich zdech sledujeme mnohé odkazy na komentáře, videa a články, formující se názory jedinců i davů, vznikající i zanikající politické myšlenky. Nakolik jsou tato samozvaná masmédia relevantní a kde mají své nedostatky, obzvláště v rovnoměrném šíření informací napříč společností, se budeme zabývat dále.

\subsection{Filter bubble}
\noindent \textit{Filter bubble} nebo také \textit{informační bublina} je jeden z mnoha fenoménů dnešní doby. Jako první na ni upozornil a popsal ji Eli Pariser~\cite{Pariser2011, PariserTed}. Jde o jev vyskytující se na sociálních sítích, kdy uživatel každou online aktivitou\footnote{Tím rozumějmě kliknutí, sdílení, komentování a obdobné činnosti.} poukazuje na oblasti jeho zájmů. Tyto algoritmy snažící se usnadnit život pomocí personalizace viděných informací se každou další činností uživatele zdokonalují, což v konečném výsledku znamená, že jsou to právě ony, jež rozhdují, co bude pro uživatlele viditelné a co naopak konsekvencí prací algoritmů bude uživateli podáno v menší míře, či úplně skryto v celkovém proudu informací~\cite{TheImpactOfFilterBubble}. Celkově to tedy může vést jedince do situace, kdy místo širokého spektra příspěvků na sobě obsahově nezávislých, vidí příspěvky jen takové, jež byly vybrány preferenčními algoritmy na základě jeho předešlé činnosti a tedy velmi zúžené škále informací.

Vezmeme-li v úvahu, kolik času lidé tráví na sociálních sítích, docházíme k závěru, že jejich názory a postoje se primárně vytvářejí zrovna zde~\cite{TheImpactOfFilterBubble, BeyondFilterBubble}.  Sleduje-li uživatel pouze názorově shodné příspěvky může toV demokratických společnostech by Filter bubble mohla představovat značné ohrožení systému, neboť  uživatelům předkládá již vyfiltrované příspěvky a to zejména takové, jež by podpořili názor uživatele samotného, nikoli




Potřebujeme popsat jak funguje twitter (ze je tam omezeny pocet slov, jak funguje following a podobne)~\cite{wikiTwitter}

a taky proc je twitter relevantni ke studiu~\cite{whyNewsOnTwitter}

\newpage
\section{Data a metody (nematematicky)}
\noindent Jak jsme již v předchozích kapitolách zmínili, s pokrokem v oblasti technologie probíhá i obrovský pokrok v oblasti čerpání zpráv a informací. Narozdíl od minulosti, drtivá většina obyvatel vyspělých států má přístup k novinkám v průběhu celého dne. Denní počet příspěvků na \textit{Twitteru} dosahoval v roce 2010 neskutečných 6 milionů~\cite{Mathioudakis2010}. Je zřejmé, že není možné používat stejné postupy pro analýzu takto rozsáhlého množství dat jako dříve. Naštěstí se v již nějakou dobu významným tempem posouvá i oblast \textit{Big Data, Machine Learningu} a jejich podoblastí. Tyto obory nacházejí uplatnění téměř ve všech oblastech dnešní vědy~\cite{Huberman2012-2-15} a stále větší pozornosti se jim dostává i v sociologii~\cite{Tinati2014, McFarland2016, Shah2015-04-09}.

\subsection{Twitter data}
\noindent Právě data z Twitteru jsou pro nás velmi vhodná. Jak se ukazuje~\cite{whyNotFb}, narozdíl on \textit{Facebooku}, Twitter užívá mnoho lidí jako zdroj informací a zpráv o aktuálním dění ve světě. Samozřejmě můžeme zpochybňovat validitu a přesnost informací, které se na takovýchto sítích objěvují. Ať už jsou takovéto obavy oprávněné, nebo ne, vůzkum sociologických jevů, který v této práci provádíme to nijak neovlivnuje.

Dalším důležitým důvodem, který vedl k výběru Twitteru jako média, ze kterého budeme stahovat informace je snadná přístupnost k datům pomocí služby API poskytovaný právě přímo Twitterem~\cite{twitterAPI}. Konkrétně pro naše účely jsme tuto službu nepoužívali přímo, ale za pomocí balíčku \textit{tweepy}~\cite{tweepy} pro programovací jazyk \textit{python}. Ten umožňuje velmi snadné ovládání a filtrování proudu dat, které si vyžádáme Twitteru a také jejich okamžitou analýzu a zpracování v pythonu.

\section{Sentimentální analýza textu}
\noindent


\newpage
\section{Závěr}
\noindent

\newpage
\begin{thebibliography}{99}

    \bibitem{BeyondFilterBubble}
    LIAO, Q. Vera a Wai-Tat FU. Beyond the filter bubble: Interactive effects of perceived threat and topic involvement on selective exposure to information. \textit{Proceedings of the SIGCHI Conference on Human Factors in Computing Systems - CHI '13}. New York, New York, USA: ACM Press, 2013, 2359-2368. DOI: 10.1145/2470654.2481326. ISBN 9781450318990. Dostupné také z: \url{http://dl.acm.org/citation.cfm?doid=2470654.2481326}

    \bibitem{TheImpactOfFilterBubble}
    GOTTRON, Thomas a Felix SCHWAGEREIT. The Impact of the Filter Bubble -- A Simulation Based Framework for Measuring Personalisation Macro Effects in Online Communities. \textit{ARXIV: Computer Science - Social and Information Networks}. 2016, \textbf{2016}. 2016arXiv161206551G. Dostupné také z: \url{https://arxiv.org/abs/1612.06551\#}

    \bibitem{PariserTed}
    PARISER, Eli. (2011). \textit{Beware online "filter bubbles"} [online]. Dostupné z \url{https://www.ted.com/talks/eli_pariser_beware_online_filter_bubbles}

    \bibitem{Pariser2011}
    PARISER, Eli. \textit{The filter bubble: what the Internet is hiding from you}. New York: Penguin Press, 2011. ISBN 15-942-0300-8.

    \bibitem{wikiTwitter}
    Twitter. In: \textit{Wikipedia: the free encyclopedia} [online]. San Francisco (CA): Wikimedia Foundation, 2017, 2017-01-22 [cit. 2017-01-22]. Dostupné z: \url{https://en.wikipedia.org/wiki/Twitter}

    \bibitem{whyNotFb}
    BARTHEL, MICHAEL, ELISA SHEARER, JEFFREY GOTTFRIED a AMY MITCHELL. \textit{The Evolving Role of News on Twitter and Facebook} [online]. In: . Pew Research Center, 2015 [cit. 2017-01-22]. Dostupné z: \url{http://www.journalism.org/2015/07/14/the-evolving-role-of-news-on-twitter-and-facebook/}

    \bibitem{whyNewsOnTwitter}
    Twitter and the News: How people use the social network to learn about the world. \textit{American Press Institute: Insights, tools and research to advance journalism} [online]. 2015, \textbf{2015} [cit. 2017-01-22]. Dostupné z: \url{https://www.americanpressinstitute.org/publications/reports/survey-research/how-people-use-twitter-news/}

    \bibitem{Mathioudakis2010}
    MATHIOUDAKIS, Michael a Nick KOUDAS. TwitterMonitor. \textit{Proceedings of the 2010 international conference on Management of data - SIGMOD '10}. New York, New York, USA: ACM Press, 2010, \textbf{2010}, 1155-1158. DOI: 10.1145/1807167.1807306. ISBN 9781450300322. Dostupné také z: \url{http://portal.acm.org/citation.cfm?doid=1807167.1807306}

    \bibitem{Tinati2014} % big data in sociology
    TINATI, Ramine, Susan HALFORD, Leslie CARR a Catherine POPE. Big Data: Methodological Challenges and Approaches for Sociological Analysis. \textit{Sociology}. 2014, \textbf{48}(4), 663-681. DOI: 10.1177/0038038513511561. ISSN 0038-0385. Dostupné také z: \url{http://journals.sagepub.com/doi/10.1177/0038038513511561}
    \bibitem{McFarland2016} % big data in sociology
    MCFARLAND, Daniel A., Kevin LEWIS a Amir GOLDBERG. Sociology in the Era of Big Data: The Ascent of Forensic Social Science. \textit{The American Sociologist}. 2016, \textbf{47}(1), 12-35. DOI: 10.1007/s12108-015-9291-8. ISSN 0003-1232. Dostupné také z: \url{http://link.springer.com/10.1007/s12108-015-9291-8}
    \bibitem{Shah2015-04-09} % big data in sociology
    SHAH, D. V., J. N. CAPPELLA a W. R. NEUMAN. Big Data, Digital Media, and Computational Social Science: Possibilities and Perils. \textit{The ANNALS of the American Academy of Political and Social Science}. 2015, \textbf{659}(1), 6-13. DOI: 10.1177/0002716215572084. ISSN 0002-7162. Dostupné také z: \url{http://ann.sagepub.com/cgi/doi/10.1177/0002716215572084}
    \bibitem{Huberman2012-2-15} % big data anywhere
    HUBERMAN, Bernardo A. Sociology of science: Big data deserve a bigger audience. \textit{Nature}. 2012-2-15, \textbf{482}(7385), 308-308. DOI: 10.1038/482308d. ISSN 0028-0836. Dostupné také z: \url{http://www.nature.com/doifinder/10.1038/482308d}

    \bibitem{tweepy}
    HILL, Aaron a Joshua ROESSLEIN. Tweepy. \textit{Github} [online]. [cit. 2017-01-22]. Dostupné z: \url{https://github.com/tweepy/tweepy}
    \bibitem{twitterAPI}
    API Overview. \textit{Twitter Developer Documentation} [online]. Twitter, 2016 [cit. 2017-01-22]. Dostupné z: \url{https://dev.twitter.com/overview/api}



\end{thebibliography}
%%%%%%%%%%%%ZKONTROLUJ \TEXTBF V CITACICH%%%%%%%%%%%%%%%%%%%%%%

\newpage
%%% The Appendix.
\section*{Přílohy}
\addcontentsline{toc}{section}{Přílohy}
\subsection*{Prvni kapitola přiloh}
\noindent
\end{document}
