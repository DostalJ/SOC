%%%%% Single page layout:
\documentclass[12pt, a4paper]{article}
\usepackage[czech]{babel}
\usepackage[utf8]{inputenc}

\begin{document}
\noindent \textbf{Název práce}: \#filterbubble\\
\textbf{Autoři:} Františka Sandroni, Jakub Dost\'al\\
\textbf{Škola}: Slovanské gymnázium Olomouc\\
\textbf{Abstrakt:}\\
Významným zdrojem informací v dnešní době jsou sociální sítě, jejich obsah je však ovlivěn \textit{preferenčními algoritmy}. Ty filtrují informace a~vedou jedince do situace, kdy je uzavřen v názorově homogenní bublině. Tomuto fenoménu se říká \textit{filter bubble} a~může zapříčinit například samovolný vznik extrémistických názorů a~skupin.

V naší práci představujeme nový způsob studia informační bubliny na kon\-krét\-ních online komunitách. Na rozdíl od mnoha předešlých studií nám umožňuje výzkum přímo v místě vzniku \textit{informační bubliny}, tj. na sociálních sítích. Měření je založeno na \textit{sentimentální analýze} příspěvků viditelných studovanými uživateli na \textit{Twitteru}. To nám umožňuje odhadovat velmi přesně složení příspěvků v okolí velkého množství námi zvolených uživatelů.

Na několika příkladech jsme ukázali její funkčnost a schopnost podávat relevantní sociologické výsledky.

Naší prací se snažíme položit základní kámen studia \textit{informační bubliny} pomocí matematiky, založeného na skutečných datech. Snažíme se otevřít dveře dalším i analytickým pohledům na tuto problematiku.\\
\textbf{Klíčová slova}: informační bublina, sociální sítě, Twitter, sentimentální a\-na\-lý\-za, společnost

\end{document}
