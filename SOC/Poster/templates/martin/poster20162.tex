%%%%%%%%%%%%%%%%%%%%%%%%%%%%%%%%%%%%%%%%%
% baposter Landscape Poster
% LaTeX Template
% Version 1.0 (11/06/13)
%
% baposter Class Created by:
% Brian Amberg (baposter@brian-amberg.de)
%
% This template has been downloaded from:
% http://www.LaTeXTemplates.com
%
% License:
% CC BY-NC-SA 3.0 (http://creativecommons.org/licenses/by-nc-sa/3.0/)
%
%%%%%%%%%%%%%%%%%%%%%%%%%%%%%%%%%%%%%%%%%

%----------------------------------------------------------------------------------------
%	PACKAGES AND OTHER DOCUMENT CONFIGURATIONS
%----------------------------------------------------------------------------------------

\documentclass[landscape,a0paper,fontscale=0.285]{baposter} % Adjust the font scale/size here

\usepackage{graphicx} % Required for including images
\graphicspath{{figures/}} % Directory in which figures are stored

\usepackage{amsmath} % For typesetting math
\usepackage{amssymb} % Adds new symbols to be used in math mode
%\usepackage[czech]{babel}
\usepackage[utf8x]{inputenc}
\usepackage{psfrag,graphicx}
\usepackage{fancybox,color,colortbl}
\usepackage{float, picinpar}
\usepackage{multicol}
\usepackage{amsmath}
\usepackage{epstopdf}

\usepackage{booktabs} % Top and bottom rules for tables
\usepackage{enumitem} % Used to reduce itemize/enumerate spacing
\usepackage{palatino} % Use the Palatino font
\usepackage[font=small,labelfont=bf]{caption} % Required for specifying captions to tables and figures

\usepackage{multicol} % Required for multiple columns
\setlength{\columnsep}{1.5em} % Slightly increase the space between columns
\setlength{\columnseprule}{0mm} % No horizontal rule between columns

\usepackage{tikz} % Required for flow chart
\usetikzlibrary{shapes,arrows} % Tikz libraries required for the flow chart in the template

\newcommand{\compresslist}{ % Define a command to reduce spacing within itemize/enumerate environments, this is used right after \begin{itemize} or \begin{enumerate}
\setlength{\itemsep}{1pt}
\setlength{\parskip}{0pt}
\setlength{\parsep}{0pt}
}

\definecolor{lightblue}{rgb}{0.145,0.6666,1} % Defines the color used for content box headers

\begin{document}

\begin{poster}
{
headerborder=closed, % Adds a border around the header of content boxes
colspacing=1em, % Column spacing
bgColorOne=white, % Background color for the gradient on the left side of the poster
bgColorTwo=white, % Background color for the gradient on the right side of the poster
borderColor=lightblue, % Border color
headerColorOne=black, % Background color for the header in the content boxes (left side)
headerColorTwo=lightblue, % Background color for the header in the content boxes (right side)
headerFontColor=white, % Text color for the header text in the content boxes
boxColorOne=white, % Background color of the content boxes
textborder=roundedleft, % Format of the border around content boxes, can be: none, bars, coils, triangles, rectangle, rounded, roundedsmall, roundedright or faded
eyecatcher=true, % Set to false for ignoring the left logo in the title and move the title left
headerheight=0.1\textheight, % Height of the header
headershape=roundedright, % Specify the rounded corner in the content box headers, can be: rectangle, small-rounded, roundedright, roundedleft or rounded
headerfont=\Large\bf\textsc, % Large, bold and sans serif font in the headers of content boxes
%textfont={\setlength{\parindent}{1.5em}}, % Uncomment for paragraph indentation
linewidth=2pt % Width of the border lines around content boxes
}
%----------------------------------------------------------------------------------------
%	TITLE SECTION 
%----------------------------------------------------------------------------------------
%
{\includegraphics[height=2cm]{sgo.png}} % First university/lab logo on the left
{\bf\textsc{Kompoziční tabulky vs. log-lineární model}\vspace{0.5em}} % Poster title
{\textsc{\{ Martin Vondrák\} \hspace{12pt} Slovanské gymnáziun Olomouc}} % Author names and institution
{\includegraphics[height=2cm]{up.png}} % Second university/lab logo on the right

%----------------------------------------------------------------------------------------
%	OBJECTIVES
%----------------------------------------------------------------------------------------
\headerbox{Log-lineární modely}{name=log-linear,column=0,span=1,row=0}{
%V případě $n$ pozorování hodnot dvou veličin můžeme jejich realizace zapsat ve tvaru čtyřpolní kontigenční tabulky
Čtyřpolní kontigenční tabulky
\begin{center}
%\vspace{0.5cm}
\begin{tabular}{c|c c|c}
&1&2&\\
\hline
1&$\pi_{11}$&$\pi_{12}$&$\pi_{1\cdot}$\\
2&$\pi_{21}$&$\pi_{22}$&$\pi_{2\cdot}$\\
\hline
&$\pi_{\cdot1}$&$\pi_{\cdot2}$&1\\
\end{tabular}
$\Rightarrow$
\begin{tabular}{c|c c|c}
&1&2&\\
\hline
1&$n_{11}$&$n_{12}$&$n_{1\cdot}$\\
2&$n_{21}$&$n_{22}$&$n_{2\cdot}$\\
\hline
&$n_{\cdot1}$&$n_{\cdot2}$&$n$\\
\end{tabular}
\end{center}
%Při nezávislosti znaků $X,Y$ obdržíme

%\begin{enumerate}
%\item $\pi_{11}={\sf P}(X=1, Y=1)= {\sf P}(X=1)\cdot{\sf P}(Y=1)=\pi_{1\cdot}\pi_{\cdot1},$
%\item $\pi_{12}={\sf P}(X=1, Y=2)= {\sf P}(X=1)\cdot{\sf P}(Y=2)=\pi_{1\cdot}\pi_{\cdot2},$
%\item $\pi_{21}={\sf P}(X=2, Y=1)= {\sf P}(X=2)\cdot{\sf P}(Y=1)=\pi_{2\cdot}\pi_{\cdot1},$
%\item $\pi_{22}={\sf P}(X=2, Y=2)= {\sf P}(X=2)\cdot{\sf P}(Y=2)=\pi_{2\cdot}\pi_{\cdot2}.$
%\end{enumerate}
Ze série čtyřpolních kontingenčních tabulek lze zachytit průměrné chování hodnot v tabulkách pomocí nezávislého log-lineárního modelu:
\begin{displaymath}
\ln x_{ij}=\beta_{0}+\beta_{1}I_{2rad}+\beta_{2}I_{2sloup}(+\beta_{3}I_{2rad}I_{2sloup})
\end{displaymath}
\begin{itemize}
\item $\ln x_{11}=\beta_{0},$
\item $\ln x_{12}=\beta_{0}+\beta_{2},$
\item $\ln x_{21}=\beta_{0}+\beta_{1},$
\item $\ln x_{22}=\beta_{0}+\beta_{1}+\beta_{2}(+\beta_{3}$)
\end{itemize}

Po odlogaritmování

\begin{itemize}
\item $x_{11}=e^{\beta_{0}},$
\item $x_{12}=e^{\beta_{0}}e^{\beta_{2}},$
\item $x_{21}=e^{\beta_{0}}e^{\beta_{1}},$
\item $x_{22}=e^{\beta_{0}}e^{\beta_{1}}e^{\beta_{2}}(e^{\beta_{3}}).$
\end{itemize}
}
%\headerbox{Kompozièní data, Aitchisonova geometrie}{name=kompozice,column=1,span=2,row=0}{
%Zkoumání vztahù mezi statistickými znaky je jeden z nejèastìjších úkolù statistické analýzy. V~poslední dobì se rozvíjí testování vztahù mezi promìnnými v tabulce pomocí tzv. logratio metodiky %kompozièních dat.
%Obecnì, za D-složkovou kompozici považujeme øádkový vektor $\textbf{x}=(x_1,x_2,...,x_D)$, kde jednotlivé složky vyjadøují relativní pøíspìvky èásti na celku.
%Výbìrový prostor potom takovéto kompozice se nazývá simplex: 
%\begin{displaymath}
%S^D=\left\{\textbf{x}=(x_1,x_2,...,x_D)|x_i>0,i=1,2,..,D; \sum\limits_{i=1}^D x_{i}=\kappa \right\}.
%\end{displaymath}
%Simplex je prostor do kterého promítneme kompozièní data pøi zachování pomìrù mezi jejich složkami. Takto promítnutá data potom mají konstatní souèet $\kappa$. V praxi to znamená, že na kompozice %aplikujeme opraci \textit{uzávìru kompozice},
%\begin{displaymath} 
%C{(\textbf{x})}=\left(\frac{\kappa \cdot x_{1}}{\sum\nolimits_{i=1}^{D} x_i},\frac{\kappa \cdot x_{2}}{\sum\nolimits_{i=1}^{D} x_i},...,\frac{\kappa \cdot x_{D}}{\sum\nolimits_{i=1}^{D} x_i}\right)
%\end{displaymath}
%}
\headerbox{Simulační studie}{name=simulace,,column=2,span=1,below=sourlog}{

%\begin{figure}[htpb]
%\begin{center}
%\begin{minipage}{7cm}{
%\vspace{-5mm}
%\includegraphics[width=\textwidth]{PCrad.eps}
%}\end{minipage} \hspace{10mm}
%\begin{minipage}{7cm}{
%\vspace{-5mm}
%\includegraphics[width=\textwidth]{PCsloup.eps}
%}\end{minipage} \\[5mm]
%\begin{minipage}{7cm}{
%\vspace{-5mm}
%\includegraphics[width=\textwidth]{PCint.eps}
%}\end{minipage} \hspace{10mm}
%\caption{Hodnoty jednotlivých souřadnic pro tabulky PC vs. prospěšnost času.}
%\end{center}
%\end{figure}
Rozdělte 100 \% svého času mezi čtyři pole, které značí, jestli jste byli pod vlivem kofeinu a jestli svůj volný čas vnímáte jako prospěšný, nebo neprospěšný.
\begin{center}
\begin{tabular}{|c|c|c|}
\hline
čas\textbackslash kofein&S kofeinu&Bez kofeinu\\\hline
Prospěšně&&\\\hline
Neprospěšně&&\\\hline
\end{tabular}
\end{center}
$$
\begin{array}{cc}
\begin{minipage}{0.3\textwidth}
\includegraphics[width=1\textwidth]{Koffeinrad.eps}
\end{minipage}
&
\begin{minipage}{0.3\textwidth}
\includegraphics[width=1\textwidth]{Koffeinsloup.eps}
\end{minipage}
\end{array}
$$
$$
\begin{array}{c}
\begin{minipage}{0.3\textwidth}
\includegraphics[width=1\textwidth]{Koffeinint.eps}
\end{minipage}
\end{array}
$$
\begin{center}
\begin{tabular}{ccc}
$\mathbf{z}_{rad}=0,6803167$&$\mathbf{z}_{sloup}=-0,1287121$\\
$\textit{sd}_{rad}=0,5855648$&$\textit{sd}_{sloup}=0,9331682$\\
$\exp(\mathbf{z}_{rad})=1,974503$&$\exp(\mathbf{z}_{sloup})=0,8792271$\\
\end{tabular}

\vspace{0.25cm}

\begin{tabular}{c}
$\mathbf{z}_{int}=0,1233194$\\
$\textit{sd}_{int}=0,71225$\\
$\exp(2\times\mathbf{z}_{int})=1,279717$\\
\end{tabular}

\end{center}
}
%%%%%%%%%%%%%%%%%%%%%%%%%%%%%%%%%%%%%%%%%%%%%%%%%%%%%%%%%%%%%%%%%%%%%%%%%%%%%%
\headerbox{Simulační studie}{name=lace,column=3,span=1,below=sourlog}{
Rozdělte 100 \% svého času mezi čtyři pole, které značí, jak svůj čas rozdělujete mezi čas na PC (mobilu, tabletu, apod.) a jestli svůj volný čas vnímáte jako prospěšný, nebo neprospěšný.
\begin{center}
\begin{tabular}{|c|c|c|}
\hline
čas\textbackslash PC&Na PC &Mimo PC\\\hline
Prospěšně&&\\\hline
Neprospěšně&&\\\hline
\end{tabular}
\end{center}
$$
\begin{array}{cc}
\begin{minipage}{0.3\textwidth}
\includegraphics[width=1\textwidth]{PCrad.eps}
\end{minipage}
&
\begin{minipage}{0.3\textwidth}
\includegraphics[width=1\textwidth]{PCsloup.eps}
\end{minipage}
\end{array}
$$
$$
\begin{array}{c}
\begin{minipage}{0.3\textwidth}
\includegraphics[width=1\textwidth]{PCint.eps}
\end{minipage}
\end{array}
$$
\begin{center}
\begin{tabular}{cc}
$\mathbf{z}_{rad}=0,3886242$&$\mathbf{z}_{sloup}=0,3162596$\\
$\textsl{sd}_{rad}=0,6726396$&$\textsl{sd}_{sloup}=0,6271583$\\
$\exp(\mathbf{z}_{rad})=1,47495$&$\exp(\mathbf{z}_{sloup})=1,371986$\\
\end{tabular}

\vspace{0.25cm}

\begin{tabular}{c}
$\mathbf{z}_{int}=-0,1567522$\\
$\textsl{sd}_{int}=0,8782147$\\
$\exp(2\times\mathbf{z}_{int})=0,7308812$\\
\end{tabular}
\end{center}
%\vspace{0,2cm}

%1: $\ln x=5+1,1I_{2rad}-2,1I_{2sloup}-0,5I_{2rad}I_{2sloup}+chyba$
%\vspace{0.2cm}

%2: $\ln x=6+2,3I_{2rad}+3,5I_{2sloup}+4,5I_{2rad}I_{2sloup}+chyba$

%\vspace{0.2cm}

%3: $\ln x=12-18I_{2rad}+10I_{2sloup}+15I_{2rad}I_{2sloup}+chyba$

%\vspace{0.2cm}
%\begin{itemize}
%Uvažovali jsme, jak nezávislý model, tak model s interakcemi
%\vspace{0.2cm}

%Rozsah tabulek: 50, 200 a 500
%\vspace{0.2cm}

%Nagenerovaná chyba z normálního rozdělení přičtena ke každé hodnotě v tabulkách
%\vspace{0.2cm}

%Rozptyly nagenerovaných chyb: 1, 3, 5
%\vspace{0.2cm}


%\end{itemize}
}
\headerbox{Kompoziční tabulky}{name=ktabulky,column=1,span=1,row=0}{
Čtyřpolní kompoziční tabulku dostaneme po uspořádaní čtyřsložkové kompozice do tabulky
$$
\textbf{x}=C\left( \begin{array}{cc}
x_{11} & x_{12} \\
x_{21} & x_{22} \\
\end{array}\right)
$$
Dále ji můžeme rozložit díky Aitchisonově geometrii na její nezávislou a interakční tabulku pomocí vztahu:
\begin{displaymath}
\mathbf{x}=\mathbf{x}_{ind}\oplus \mathbf{x}_{int}
\end{displaymath}
\begin{center}
Nezávislá tabulka 
\end{center}
$$
\textbf{x}_{ind}=\left( \begin{array}{cc}
x_{11}\sqrt{x_{12}x_{21}} & x_{12}\sqrt{x_{11}x_{22}} \\
x_{21}\sqrt{x_{11}x_{22}} & x_{22}\sqrt{x_{12}x_{21}} \\
\end{array}\right)
$$
\begin{center}
Interakční tabulka 
\end{center}
$$
\textbf{x}_{int}=\left( \begin{array}{cc}
\sqrt{x_{11}x_{22}} & \sqrt{x_{12}x_{21}} \\
\sqrt{x_{12}x_{21}} & \sqrt{x_{11}x_{22}} \\
\end{array}\right)
$$
\vspace{0.25cm}
}
\headerbox{Souřadnice}{name=souradnice,column=2,span=1,row=0}{
Souřadnice řádku:
\begin{displaymath}
\mathbf{z}_{rad}=\ln\frac{\sqrt{g(x_{1\cdot})}}{\sqrt{g(x_{2\cdot}}}=\frac{1}{2}\ln\frac{x_{11}x_{12}}{x_{21}x_{22}}
\end{displaymath}
Souřadnice sloupce:
\begin{displaymath}
\mathbf{z}_{sloup}=\ln\frac{\sqrt{g(x_{\cdot1})}}{\sqrt{g(x_{\cdot2}}}=\frac{1}{2}\ln\frac{x_{11}x_{21}}{x_{12}x_{22}}
\end{displaymath}
Interakční souřadnice:
\begin{displaymath}
\mathbf{z}_{int}=\frac{1}{2}\ln\frac{x_{11}x_{22}}{x_{12}x_{21}}
\end{displaymath}
}
\headerbox{Vyjádření souřadnic pomocí log-lineárního modelu}{name=sourlog,column=3,span=1,row=0}{
Dosazením výpočtu četností v tabulce z log-lineárního modelu do souřadnic pro kompoziční tabulky a následné úpravy dostaneme:

\vspace{0.5cm}

$\beta_{3}=0$
\begin{displaymath}
\sqrt{\frac{x_{11}x_{12}}{x_{21}x_{22}}}=e^{-\beta_{1}}\,\,\,\sqrt{\frac{x_{11}x_{21}}{x_{12}x_{22}}}=e^{-\beta_{2}}\,\,\,\frac{x_{11}x_{22}}{x_{12}x_{21}}=1
\end{displaymath}
%\begin{displaymath}
%\sqrt{\frac{x_{11}x_{21}}{x_{12}x_{22}}}=e^{-\beta_{2}}\,\,\,\
%\end{displaymath}
%\begin{displaymath}
%\frac{x_{11}x_{22}}{x_{12}x_{21}}=1
%\end{displaymath}

\vspace{0.5cm}

$\beta_{3}\neq 0$
\begin{displaymath}
\sqrt{\frac{x_{11}x_{12}}{x_{21}x_{22}}}=e^{-\beta_{1}-\beta_{3}/2}\,\,\,\sqrt{\frac{x_{11}x_{21}}{x_{12}x_{22}}}=e^{-\beta_{2}-\beta_{3}/2}
\end{displaymath}
%\begin{displaymath}
%\sqrt{\frac{x_{11}x_{21}}{x_{12}x_{22}}}=e^{-\beta_{2}-\beta_{3}/2}\,\,\,\
%\end{displaymath}
\begin{displaymath}
\sqrt{\frac{x_{11}x_{22}}{x_{12}x_{21}}}=e^{\beta_{3}/2}
\end{displaymath}
Po úpravě získáme hodnoty parametrů log-lineárních modelů:
\begin{displaymath}
\beta_{1}=-\frac{1}{2}\ln\frac{x_{11}x_{21}}{x_{12}x_{22}}-\frac{1}{2}\ln\frac{x_{11}x_{22}}{x_{12}x_{21}}
\end{displaymath}
\begin{displaymath}
\beta_{2}=-\frac{1}{2}\ln\frac{x_{11}x_{22}}{x_{12}x_{21}}-\frac{1}{2}\ln\frac{x_{11}x_{22}}{x_{12}x_{21}}
\end{displaymath}
\begin{displaymath}
\beta_{3}=2\frac{1}{2}\ln\frac{x_{11}x_{22}}{x_{12}x_{21}}
\end{displaymath}
}
\headerbox{Volba souřadnic}{name=volba,column=1,span=1,below=ktabulky, above=bottom}{
Postupné binární dìlení (PBD) v kompozièních tabulkách
\begin{center}
\begin{tabular}{c c c c c c c}
\hline
PBD&$x_{11}$&$x_{12}$&$x_{21}$&$x_{22}$&r&s\\\hline
Krok 1&+&-&-&+&2&2\\
Krok 2&&+&-&&1&1\\
Krok 3&+&&&-&1&1\\
\hline
\end{tabular}
\end{center}
\begin{displaymath}
z_{i} = \sqrt{\frac{rs}{r+s}}\ln\frac{(x_{j1}
\cdot x_{j2}\cdots  x_{jr} )^{1/r}}{(x_{k1}\cdot x_{k2}\cdots x_{ks})^{1/s}},
\end{displaymath}
Souøadnice vzhledem k Aitchisonovì geometrii
\begin{center}
\begin{tabular}{|c|ccc|}
\hline
&$\mathbf{x}$&$\mathbf{x}_{ind}$&$\mathbf{x}_{int}$\\\hline
$\mathbf{z}_{1}$&$\frac{1}{2}\mathrm{\ln}\frac{x_{11}x_{22}}{x_{12}x_{21}}$&$0$&$\frac{1}{2}\mathrm{\ln}\frac{x_{11}x_{22}}{x_{12}x_{21}}$\\
$\mathbf{z}_{2}$&$\frac{\sqrt{2}}{2}\mathrm{\ln}\frac{x_{12}}{x_{21}}$&$\frac{\sqrt{2}}{2}\mathrm{\ln}\frac{x_{12}}{x_{21}}$&$0$\\
$\mathbf{z}_{3}$&$\frac{\sqrt{2}}{2}\mathrm{\ln}\frac{x_{11}}{x_{22}}$&$\frac{\sqrt{2}}{2}\mathrm{\ln}\frac{x_{11}}{x_{22}}$&$0$\\[1mm]
\hline
\end{tabular}
\end{center}
%Souřadnice řádku:
%\begin{displaymath}
%\mathbf{z}_{rad}=\ln\frac{\sqrt{g(x_{1\cdot})}}{\sqrt{g(x_{2\cdot}}}=\frac{1}{2}\ln\frac{x_{11}x_{12}}{x_{21}x_{22}}
%\end{displaymath}
%Souřadnice sloupce:
%\begin{displaymath}
%\mathbf{z}_{sloup}=\ln\frac{\sqrt{g(x_{\cdot1})}}{\sqrt{g(x_{\cdot2}}}=\frac{1}{2}\ln\frac{x_{11}x_{21}}{x_{12}x_{22}}
%\end{displaymath}
%Interakční souřadnice:
%\begin{displaymath}
%\mathbf{z}_{int}=\frac{1}{2}\ln\frac{x_{11}x_{22}}{x_{12}x_{21}}
%\end{displaymath}
}
\headerbox{Kontakt}{name=kontakt,column=0,span=1,below=kontigencni, above=bottom}{
Martin Vondrák

\vspace{0,1cm}

vondrakmar@seznam.cz

\vspace{0.1cm}

+420 737 778 448

\vspace{0,1cm}

}
\headerbox{Literatura}{name=literatura,column=3,span=1,row=2,below=simulace,above=bottom}{
\scriptsize{FAČEVICOVÁ, Kamila, HRON, Karel, TODOROV, Valentin, TEMPL, Matthias. 
Compositional tables analysis in coordinates. \em{Scandinavian Journal of Statistics}, 2016, přijato k tisku.

PAWLOWSKY-GLAHN, Vera, BUCCIANTI, Antonella (eds). \textit{Compositional data analysis: Theory and applications}. Chichester: Wiley, 2011. ISBN 978-0-470-71135-4

FAČEVICOVÁ, Kamila,  HRON, Karel, TODOROV, Valentin, GUO, Dong,  TEMPL, Matthias.  Logratio approach to statistical nalysis of 2$\times$2 compositional tables. \textit{Journal of Applied Statistics}, 2013,  \textbf{41}(5), 944-958. ISSN 0266-4763.
%R Core Team: R: A language and environment for statistical computing

%M. VONDRÁK: Statistická analýza nezávislosti ve ètyøpolních tabulkách dat
}}
\end{poster}

\end{document}